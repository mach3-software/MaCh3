{\bfseries{Process\+MCMC}} -\/ The main app you want to use for analysing the ND280 chain. It prints posterior distribution after burn-\/in the cut. Moreover, you can compare two/three different chains. There are a few options you can modify easily inside the app like selection, burn-\/in cut, and whether to plot xse+flux or only flux. Other functionality 
\begin{DoxyEnumerate}
\item Produce a covariance matrix with multithreading (which requires lots of RAM due to caching)  
\item Violin plots  
\item Credible intervals and regions  
\item Calculate Bayes factor and give significance based on Jeffreys scale  
\item Produce triangle plots  
\item Study covariance matrix stability  
\end{DoxyEnumerate}

{\bfseries{Get\+Penalty\+Term}} -\/ Since xsec and flux and ND spline systematic are treated as the same systematic object we cannot just take log\+\_\+xsec, hence we need this script, use {\ttfamily Get\+Flux\+Penalty\+Term MCMCChain.\+root config}. Parameters of relevance are loaded via config, thus you can study any combination you want. Time needed increases with number of sets \+:(

{\bfseries{Diag\+MCMC}} -\/ Perform MCMC diagnostic like autocorrelation or trace plots.

{\bfseries{RHat}} -\/ Performs RHat diagnostic to study if all used chains converged to the same stationary distribution. 
\begin{DoxyCode}{0}
\DoxyCodeLine{./RHat\ Ntoys\ MCMCchain\_1.root\ MCMCchain\_2.root\ MCMCchain\_3.root\ ...\ [how\ many\ you\ like]}

\end{DoxyCode}


{\bfseries{Combine\+Ma\+Ch3\+Chains}} -\/ will combine chains files produced by {\bfseries{MCMC}}, enforcing the condition that all the files to combine were made using the exact same software versions and config files 
\begin{DoxyCode}{0}
\DoxyCodeLine{CombineMaCh3Chains\ [-\/h]\ [-\/c\ [0-\/9]]\ [-\/f]\ [-\/o\ <output\ file>]\ file1.root\ [file2.root,\ file3.root\ ...]}

\end{DoxyCode}
 {\itshape file\+X.\+root} are the individual spline files to combine, can specify any number, need at least one

-\/c target compression level for the combined file, default is 1, in line with hadd

-\/f force overwrite of the combined file if it exists already

-\/h print usage message and exit

{\itshape Output file} (optional) name of the combined file. If not specified, will just use {\itshape file1.\+root}, the first in the list of files, same as {\itshape hadd}. 